\documentclass[10pt, a4paper]{article}
\usepackage[T1]{fontenc}
\usepackage[utf8]{inputenc}
\usepackage{ifpdf}
\usepackage{url}

\title{Cognition, Report exercise 1}
\date{}
\author{Oscar Olsson, 900311-0690\\ Tommy Olofsson, 900814-4553\\
	Erik Westrup, 901021-1192\\ Erik Jansson, 901202-4114}
\begin{document}
\maketitle
\begin{center}
Group: 4 \\
Teacher: Thomas Strandberg.
\end{center}
\newpage

\section{Preamble}
The purpose of this laboratory exercise was to familiarize us with a variety of cognitive phenomena. We were subject to a number of demonstrations and active experiments during the laboratory session. Afterwards we delved the subjects deeper on our own and wrote this report.

\section{Exercise 1, Inattention \& change blindness}
\subsection{Description}
To illustrate \emph{change blindness} we watched a video clip\footnote{colour changing card trick, \url{http://www.youtube.com/watch?v=voAntzB7EwE}, 2011-11-01} showing a man and a woman behind a desk, performing a card trick with a blue backed deck. The woman chose a random card from the deck and showed it to the camera, stating its color and number out loud. She then gave it back to the man, who put it back in the deck. The man then spread the deck out, face up, on the table and picked out the previously chosen card. When the man then turned the deck over, all the other cards had a red back except for the one the woman had chosen. What also changed was the color of the background cape, both persons shirts and the table-cloth. These changes were coordinated with the camera movement so that the table-cloth could quickly be changed when the camera focused right above the table on the card etc. While the audience focused on following the changes of card colors they missed all the other changes even though they are quite apparent.

To illustrate inattention blindness another video with two basket teams was shown. We were told to count the number of passes from the white team. During the video a person in a black gorilla suit walked past the scene. While focusing on the count some missed the gorilla.

\subsection{Discussion}
In this exercise we observed the phenomena called \emph{change blindness}. In the color changing card trick exercise we all missed the changes in the surroundings due to focusing on discovering the secret behind the card trick itself. This was surprising to us as we did not expect anything but the card trick to occur. 

Unfortunately this exercise was futile because we all had previously seen the clip in another course. Therefore we did not focus on the ball but the entrance of the gorilla. 

As discussed in “Unconscious Transference” As An Instance of Change Blindness \footnote{“Unconscious Transference” As An Instance of Change Blindness, Davis, Vanous \& Cucciare, 2008} eye witness testimonials might be more unreliable than previously considered due to these phenomena. From this we learn that describing something from memory other than what we focused on at the time can not be considered reliable. 

\section{Exercise 2, Probability assessments}
\subsection{Description}
We were given a set of multiple choice questions about probability and were asked to individually answer these. 
The questions were formulated in such a way that one of the answers just \emph{felt} appropriate.
Two of the questions were about choosing the larger city of two. The given choices were designed in such a way that one would recognize one but not the other. 

\subsection{Discussion}
Many participants chose the incorrect answer becuse they formed a mental vision of what they thought was the correct answer. 

The results of the previously mentioned questions exellently demonstrated the Fast and Frugal Heuristic model mentioned by Chase et al\footnote{"Vision of Rationality", Chase et al., Trends in cognitive sciences vol. 2 no. 6, 1998}. This models claims one are strongly influensed by recognition when making a binary choise. 

\section{Exercise 3, The McGurk Effect}
\subsection{Description}
The McGurk effect is about the way our perception of speech differ depending on visual stimuli. To illustrate it a quite simple and straight-forward experiment was performed; while viewing a video, we were first instructed to look down and just listen to the voice. We heard nothing more than a man saying \emph{ba, ba, ba, ba}. The second time the video was played, we were allowed to look up at the screen. The video was depicting the lower half of a face.  This second time it sounded, for some of us, like the man said \emph{ba, la, la, va} It turns out that we were actually hearing \emph{the same sound} but seeing different mouth movements.

\subsection{Discussion}
Although only half of the group did actually noticed the McGurk effect we can imagine the effect having some real-world significance. This might very well be the reason for people prefer video-conferencing and such when communication with friends and colleagues online e.g. Skype.

\section{Exercise 4, Memory demonstration}
\subsection{Description}
The exercise session began with the teacher handing out a small note to half of the class, containing the text \begin{quote}''To wash clothes''\end{quote} written in Swedish (\emph{"Att tvätta kläder"}). The other half of the class were left in the dark without any note at all. The teacher then began reading a text with a generic description of how to wash clothes and told us it was not that important and we should not pay that much attention to it. After this the exercise session continued with other topics and 30--45 minutes later the entire class were asked to recall as much of the text as possible and write it down.

The result was, perhaps surprisingly, that the persons who were shown the heading \emph{"To wash clothes"} generally had easier to remember more pieces of the read text.

\subsection{Discussion}
The point of this exercise was to illustrate that the presence of a heading makes it easier to memorize a piece of text since the brain can associate the text with a short sentence. This effect is even greater if the person is familiar with the subject at hand\footnote{"Headings as Memory Facilitators: The Importance of Prior Knowledge", Wilhite, 1989}, and, since washing clothes is a familiar task for most people (hopefully), the group of people that had something to relate the text to, i.e. the heading, had much easier to memorize it.

However, we did not noticed this effect in our group. Since the teacher pointed out that this shouldn't be payed much attention to, none the members of our group even tried to memorize the text.

\subsection{Exercise 5, The Stroop Effect}
\subsubsection{Description}
The Stroop effect describes the peculiar fact that people have to devote a lot more conscious thought to speak the color of a printed word compared to just reading the word. Particularly so if the word is the name of a color printed in another one, creating a mental clash.

The material for this experiment included just three cards.  Each card had about 15 words, each on a separate line, printed in different colors.  The words on the first card were all gibberish. This made it easy to determine the color with few errors and little stuttering. 
The second card was quite easy once we realized that the words were printed in the 'right' color. It reduced both the time taken to read the colors by a couple of seconds as well as the amount of errors and stuttering compared to the other cards.

The third card was the most difficult. The words were all names of different colors, but they were printed in a different one. For example, we might have had the word \emph{green} printed in red. This made it really hard to say the color of the print rather than reading the word itself.  We hesitated a lot and it took about twice the time to go through this card compared to the other cards. 

\subsubsection{Discussion}
Since we already knew about this effect we weren't surprised to a see significant difficulty when the words and the color did not match. The Stroop test can be combined with other methods to find e.g. attention disorders such as ADHD (although with questionable results). \footnote{"Stroop Interference and Attention-Deficit/Hyperactivity Disorder, Review and Meta-Analysis", Lansbergen \& Kenemans, 2007, \url{http://psycnet.apa.org/journals/neu/21/2/251.pdf}}. We have tried to find natural occurrences of the Stroop effect but failed to do so.\\[5mm]

\begin{tabular}{l l l l}
  Card & Characteristics & Mean time (s) & Mean number of errors \\
  \hline
  1 & non-words & 11.25 & 1.25 \\
  2 & colors right & 8.5 & 0.25 \\
  3 & colors wrong & 19.25 & 2.25 \\
\end{tabular}

\section{Exercise 6, Communication}
\subsection{Description}
This exercise was performed to illustrate how communication was affected by different limitations. We were given two copies of 18 figures divided on three separate pages. In one copy the figures was numbered from one through six on each page and in the other copy the figures were not numbered. One of us, the explainer, received the numbered figures and one of us, the responder, received the unnumbered figures. The explainer was asked to help the responder number the figures according to the explainer's figures as fast as possible. The responder and receiver were not allowed to see each others figures nor draw figures on paper or in the air. The task was performed three times, each with a new set of six figures, with different limitations.
The limitations were:
\begin{enumerate}
\item The responder may not speak.
\item The responder and explainer may not maintain visual contact.
\item No limitations.
\end{enumerate}

\subsection{Discussion}
We performed a slightly modified version of the exercise due to a missed detail of the instructions (you were not allowed to draw figures). In all three tests we found no problem in communication. In the first part the explainer used speech and the responder confirmed with hand gestures. In the second part the responder confirmed with speech. I the last part explainer communicated with drawing of figures and the responder used speech to confirm. Surprisingly the best result was achieved when no visual contact was allowed. This might be due to a difference in difficulties in explaining the figures.

From this experiment we can conclude that it is easier to communicate with speech with no visual contact than only visual contact. This confirms the common notion that speech is more powerful than visualizations. For example it is more important to implement a good speech feature than a video feature in a conference system.

%\bibliographystyle{IEEEtran}
%\bibliography{refs}
\end{document}
